\documentclass[a4paper]{article}

\usepackage{amsfonts}
\usepackage[english]{babel}
\usepackage[utf8]{inputenc}
\usepackage{mathtools}
\usepackage{amssymb}
\usepackage{graphicx}
\usepackage{tikz}
\usepackage{polynom}
\usepackage{amsthm}
\usepackage{arcs}
\usepackage}pifont}
\usepackage[colorinlistoftodos]{todonotes}

\title{A Mathematical Exploration of Virus Capsid Structures}

\author{Charles Zahara}

\date{\today}

\begin{document}
\maketitle

\begin{abstract}
This paper explores the mathematics behind the icosahedral virus capsid structure... more to come.
\end{abstract}

\section{Biological Background}

Things to include:
\begin{itemize}
\item Caspar and Klug history
\item helical and icosahedral shapes
\item theory of quasi-equivalence
\item T-number concept
\item P-number (maybe)
\end{itemize}

\section{Rotation Groups}
This section would discuss rotational symmetry, evaluate possible shapes and explain why the icosahedron was the initial optimum shape. 

\section{T-Numbers}

\begin{itemize}
\item Prove $T = P f^2$ where $P = h^2 + h k + k^2$ with $GCD(h,k) = 1$ and $f \in \mathbb{N} $ 
\end{itemize}

This document was published with \LaTeX{}
\end{document}