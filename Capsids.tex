\documentclass[12pt,letter]{article}

\usepackage{amsfonts}
\usepackage[english]{babel}
\usepackage[utf8]{inputenc}
\usepackage{mathtools}
\usepackage{amssymb}
\usepackage{graphicx}
\usepackage{gensymb}
\usepackage{tikz}
\usepackage{polynom}
\usepackage{amsthm}
\usepackage{arcs}
\usepackage{pifont}
\usepackage[colorinlistoftodos]{todonotes}
\usepackage{caption}
\usepackage{subfigure}
\newcommand*\rfrac[2]{{}^{#1}\!/_{#2}}

\newtheorem{definition}{Definition}
\newtheorem*{definition*}{Definition}

\DeclarePairedDelimiter\abs{\lvert}{\rvert}
\makeatletter
\let\oldabs\abs
\def\abs{\@ifstar{\oldabs}{\oldabs*}}%
% Absolute value symbol
\usepackage{setspace}
\doublespacing
\usepackage{fullpage}

\begin{document}

%Begin title block%
\title{Capsids and Stuff}
\author{Charles Zahara}
\maketitle
%End title block%

\section{Introduction}

% CITATION EXAMPE\cite[p 27]{Mannige:2009}

\paragraph{}
Mathematics has always been a beautiful field full of excitement and discovery. From purely theoretical considerations to strictly applied problems the mathematician is never without mental stimulus. Its inspirations are everywhere and one needs only to be willing to ask why and how to be able to find a fantastic problem to consider. Although the sciences sometimes lack a willingness to work together, Biology is a great realm within which to discover new questions. To find challenging mathematical problems within Biology, one does not need to be an applied mathematician with decades of experience in modeling. Nor does one need to have an extensive background in Biology and science. All that is required is to be willing to examine systems and be willing to ask questions.\\

The aim of this paper is to take the reader through the discovery of mathematical puzzles inspired by the physical world. Specifically these mathematical questions will come from the theory of spherical virus capsids. Two problems left unanswered by many of the works on the spherical virus capsids will be laid out and then multiple techniques for solving these problems will be detailed. Although this paper is centered around the mathematics of why icosahedral shells work, it aims to be readable by anyone without external materials, extensive background reading, or a 6' by 20' wall of white board to work out gaps in equations. This paper will ultimately demonstrate that unique problems in mathematics are everywhere and can often be solved with very basic tools if one is willing to apply some hard work and problem solving.\\

A secondary objective of this paper is to be an entertaining introduction to the theory of icosahedral symmetry of virus capsids. It is intended to be a type of tour guide through the development and arguments of this theory. As such, the paper begins with a review of both past and present work on virus capsids. First the pivotal works on the theory of symmetry in spherical virus capsids will be summarized. Then a few selections of the current work both biologists and mathematicians are performing in the field of virus capsids will be examined. Hopefully the reader will come away understanding not only the form and function of icosahedral capsids, but also the amazing, challenging, and ultimately fun mathematical problems that arise within Biology. \\

\section{Background} %%%%% 1. Background

\paragraph{}
Viruses have been shaping human history for millennia. From smallpox and polio to the yearly strains of flu and common cold, every generation has battled against these tiny creatures. They have wiped out civilizations and continue to kill millions every year. As such the study of their form and function remains important.

\begin{definition*}[Virus]
Viruses are infectious agents composed of nucleic acid surrounded by a protein shell, called a capsid, and in some cases a membranous envelope. \normalfont{ \cite{Campbell:2002} }
\end{definition*}
While the above definition may make viruses seem like simple little organisms, they have a multitude of unique variations. Their nucleic acid genetic material may come in the form of DNA or RNA, depending on the virus, and be either single or double stranded. They can be rod shaped, roughly spherical, or in some cases rather complex in shape. Additionally, viruses are very effective organisms. They cannot replicate or reproduce themselves, but instead inject their genetic material into a host cell. The host cell is then taken over and forced to create copies of the virus genetic material and capsid proteins. Eventually, once it is filled to capacity, the host cell bursts open and the newly created capsids are expelled and proceed to a new host to repeat the process. \cite{Campbell:2002} See PICTURE FROM CAMPBELL PAGE 329. \\

The purpose of these capsid shells are to protect the virus genetic material as they move from host cell to host cell. This makes the capsid crucial in the infection process. It also means that understanding the structure and mechanics of the capsids may allow scientists to develop ways to disrupt their construction, thus eliminating the virus's ability to travel and infect new host cells. \\

As stated previously, viruses come in a variety of shapes and sizes, defined by their capsid shells. Some are long tubes, similar to a straw, classified as helical viruses, others are roughly spherical, classified as icosahedral viruses, and others do not fall into either of the two previously mentioned categories and are classified as complex. GET FIGURE TO SHOW THESE Spherical viruses are the focus of this paper. In the late 1950's and early 1960's it was suggested that spherical viruses capsids must conform to icosahedral symmetry \cite{Crick:1956}, \cite{Caspar:1962}. Although there are still many exceptions which continue to be examined, through decades of study and advances in microscope technology the theory that spherical capsids must have icosahedral symmetry has been confirmed both visually and computationally for most cases. What is often left unexplained in papers discussing spherical viruses is the mathematical reasoning behind the theory of their shape.\\

Before the history and theory are reviewed a few key definitions and their relationship are important.

\begin{definition*}[Capsid]
A viral capsid is a protein shell made up multiple copies of one, but in some cases a few, small proteins, which encloses and protects the viral genome.
\end{definition*}
\begin{definition*}[Capsomere]
A capsomere is a repeated symmetric building unit for the viral capsid. They are made up of multiple proteins.
\end{definition*}
\begin{definition*}[Protomer]
A promoter is a single protein unit of a capsomere.
\end{definition*}

The important note is that the protomers are the smallest unit. They are single asymmetric proteins. Capsomeres are built up of multiple protomers in a symmetric fashion. There can be multiple differently shaped capsomeres made from the same protomer sub-unit. The Capsid is the final shell, made up of the capsomere building blocks.



\subsection{The History of Icosahedral Symmetry} %%%%% 2.1 History

\paragraph{}
In 1956 Crick and Watson wrote a paper entitled "Structure of Small Viruses." In this paper they hypothesized that "a small virus contains identical sub-units, packed together in a regular manner." This hypothesis was based upon X-ray photographs and electron micrographs of small viruses which consistently showed that viruses are very consistent in their shape. At the time, viruses either followed the rod shape of the tobacco-mosaic virus or the spherical shape of the turnip yellow mosaic virus (Tobacco studied by Tubingen and Berkley, turnip by Markham, page 473). \\

Crick and Watson's claims about spherical viruses were heavily influenced by studies of rod-shaped viruses. The tobacco mosaic virus had been shown to consist of many structurally equivalent sub-units set about a central axis in a helical array (what does this mean?). The helical nature of the protein shell meant that they central axis of the tobacco mosaic virus was a symmetry axis, suggesting that each protein sub-unit exits in an identical environment. So for each sub-unit, its interactions with its neighbors are identical to that of its neighbor. Crick and Watson then extended these observations about the rod-shaped tobacco mosaic virus to spherical viruses.


\section{T-numbers}
\paragraph{}
Caspar and Klug claim that their T-number, $T = h^2 + hk + k^2$, defines the exact number of identical asymmetric protein sub-units found in a spherical capsid through the formula $P = 60T$. Recall this 60 is because the T number measure the number of triangular capsomers??? per side of the icosahedron and each capsomer is composed of 3 of the protein sub-unit. Since an icosahedron has 12 sides, we would have $3*12*T = 60T$ proteins per capsid. While their T-number has been confirmed as accurate through better microscopes (get type and shit), as mathematicians we should ask, was this expected? Will the T-number always work for integers h and k. If we use Caspar and Klug's approach of placing one corner of a triangular face of the icosahedron at $(0,0)$ on a P6 (what does this mean) grid, will the third corner always fall on another integer valued point of the grid? To answer these questions we will take 2 different approaches. One requires no mathematics beyond high-school, and the second only requires a bit of knowledge of linear algebra.

\subsection{The geometry approach}

\paragraph{}
The goal of this approach is to establish that if we are given a point $(h,k)$ as defined by Caspar and Klug (that is that the h-axis is the standard x-axis and the k-axis is a $60\degree$ rotation counter-clockwise of the h-axis) that the area of an equilateral triangle with a base from $(0,0)$ to $(h,k)$ will equal the area of an equilateral triangle with base length 1 times $T = h^2 + hk + k^2$. We will attempt to do this with no more than the skills of someone who has passed high school geometry. \\

First let us recall that the area of an equilateral triangle is $A = \rfrac {\sqrt{3}} {4} \, b^2$ where $b$ is the length of any side of the triangle. If we did not know this formula off the top of our head, could we re-establish it? To begin, take an equilateral triangle with side length $b$ (figure). Draw a line from any corner to the midpoint of the opposing side. We now have 2 triangles. If we recall our triangle congruency theorems we can quickly see by the side-angle-side theorem that these triangles are congruent. This forces the angle we cut with our line to be bisected, and since we started with an equilateral triangle which has all angles equal to $60\degree$, we know the top angles are both $30\degree$ making our constructed line perpendicular to the base. \\

We now wish to find the length of our constructed line $l$. Thankfully the pythagorean theorem is all we need for this.
\begin{align*}
b^2 &= \left(\dfrac{b} {2}\right)^2 + l^2 \\
l^2 &= b^2 - \dfrac{b^2} {4} \\
l^2 &= \dfrac{3} {4} \, b^2 \\
l &= \sqrt{\dfrac{3} {4} \, b^2} \\
l &= \dfrac{\sqrt{3}} {2} \, b
\end{align*}
Thus using our standard triangle area formula of $A = \rfrac{1} {2} \, base * height$ we get
\begin{align*}
A &= \dfrac{1} {2} \, b * \dfrac{\sqrt{3}} {2} \, b \\
A &= \dfrac{\sqrt{3}} {4} \, b^2
\end{align*} \\

Recall our goal of establishing that the area of an equilateral triangle with a base from $(0,0)$ to $(h,k)$ will equal the area of an equilateral triangle with base length 1 times $T = h^2 + hk + k^2$. Consider the triangle created by drawing lines between $(0,0), \; (h,k) \; \text{and} \; (h,0)$. We know 2 of the side lengths and wish to find side length b. Applying our previous work, we know that the exterior angle at point $(h,0)$ is $60\degree$, so if we drop a line down from point $(h,k)$ perpendicular to our t-axis, we obtain a triangle we have already worked with and know the dimensions of. Since it is $k$ units from $(h,0)$ to $(h,k)$ we know the perpendicular we just drew has length $\rfrac{\sqrt{3}} {2} \, b$ and the final leg of the triangle is $\rfrac 1 2 \, b$.

\section{What do trapezoids have to do with it?}
This entire section is wrong. You fucked up cause you're fucking retarded.

\paragraph{}
One constant and often unexplained attribute within works discussing icosahedral virus capsids is the authors' choice of subunit shape. Almost always the subunit is represented by a trapezoid with little to no explanation of why. If we recap what we know so far:
\begin{enumerate}
	\item Capsids are made up of pentamers and hexamers
	\item Hexamers are capable of existing in a "flat" state
	\item Three subunits need to make up an equilateral triangle
\end{enumerate}
Given these conditions, especially point (3), we may simply guess that trapezoids are used because every equilateral triangle can be split into three identical (right word?) isosceles trapezoids (See INSERT figure). 

But then we must ask why not some other shape that also has this property. For instance, equilateral triangles can also be split into three identical isosceles triangles or three identical right kites. To demonstrate, extend angle bisectors from each corner of an equilateral triangle until they meet in its interior. The result is that we have split the initial equilateral triangle into three identical isosceles triangles. Or we could create perpendicular bisectors on each side of an equilateral triangle and extend them within the interior of the triangle until they meet. This will result in the triangle being split into three identical right kites. (see INSERT figure) So again we must ask, why are trapezoids the standard subunit shape if other shapes will also meet our criteria of being able to fit perfectly inside an equilateral triangle?

Upon examination of shaded and deciphered images of real world icosahedral capsids we can see that the choice of trapezoid is simply due to their real world occurrence. (See [INSERT] capsid images) This was mentioned as early as (get date) by (get guy from mannige thesis)But this is a very interesting phenomenon. As pointed out by 



\bibliographystyle{IEEEtran}
\bibliography{Capsid_Cites}



\end{document}